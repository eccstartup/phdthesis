%\phantomsection \addcontentsline{toc}{chapter}{参考文献}
%\addcontentsline{toe}{chapter}{References} \fontsize{10.5pt}{10.5pt}\selectfont
%\setlength{\baselineskip}{15pt} \addtolength{\bibsep}{-2mm}
%\bibliographystyle{setup/chinesebst1} 
%\bibliography{reference/phdthesis}

\xjtubib{reference/phdthesis}

% \xjtuappendix

% \input{body/appendice.tex}

% \xjtuendappendix

\xjtuspchapter{攻读博士期间取得的研究成果}{攻读博士期间取得的研究成
  果}{Achievements}

% 已发表或已录用的学术论文、已出版的专著/译著、已获授权的专利按参考文献格式列出。
% 科研获奖,列出格式为:获奖人(排名情况).项目名称.奖项名称及等级,发奖机构,
% 获奖时间.与学位论文相关的其它成果参照参考文献格式列出。全部研究成果连续编号编
% 排。

\renewcommand\labelenumi{[\theenumi]}
{\xiaosi \noindent 一、已发表或已录用的学术论文 \begin{enumerate}
  \begin{spacing}{1.0}
  \item Y. Lu and Y. L. Jiang. Symplectic schemes for telegraph equations[J]. Journal of Computational Mathematics, 2016, 34(3):326--340.~(国际知名~SCI~期刊)

  \item Y. L. Jiang, Y. Lu, and C. Chen. Conservation laws and optimal system of extended quantum Zakharov-Kuznetsov equation[J]. Journal of Nonlinear Mathematical Physics, 2016, 23(2): 157--166.~(国际知名~SCI~期刊, IDS~号: DQ2PO)

  \item Y. Lu, Y.L. Jiang and B. Song. Symplectic waveform relaxation methods for Hamiltonian systems[J]. Applied Mathematics and Computation, 2017, 292: 228--239.~(国际知名~SCI~期刊)
  \end{spacing}
\end{enumerate}

\vspace{12pt}

\noindent 二、科研项目
\begin{spacing}{1.0}
  \begin{enumerate}
  \item 大型并行计算的时空区域分解方法: 2011-2014, 国际科技合作项目(科技部), 参加
    者.
  \item 集成电路模拟中的数学方法研究: 2014-2017, 国家自然科学基金项目, 参加者.
  \end{enumerate}
\end{spacing}
}