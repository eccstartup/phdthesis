\chapter{结论与展望}
\echapter{Conclusions and Expectations}
本章总结前面章节的主要内容, 并为下一步的研究工作进行展望.
\section{结论}
\esection{Conclusions}
本博士论文主要围绕辛方法、波形松弛方法和李群方法开展了一些数值方法和解析方法的研究.
论文通过对不守恒的系统进行变换得到了形式较好的守恒系统,通过对隐式格式进行的修正得到了较为
容易的显式格式,通过李点对称变换得到了形式较为简单的约化方程.具体内容包括以下几个方面.

1. 提出了一种求解带有齐次边界条件的电报方程的新方法.该方法通过将非哈密尔顿系统转化为哈密尔顿系统,并利用辛方法的保结构特性,得到了良好的效果.论文对算法的阶条件、CFL 条件、长时间性质和局限性进行了讨论.该方法的基本思想是先变换,再求解,再逆变换.方法具有辛方法长时间求解的优良特性,其解可以看作一个很优美的哈密尔顿系统的解与一个函数的乘积,采用空间二阶离散格式,得到了 \texorpdfstring{$O(\Delta x^2+ \Delta t^k)$} 的误差界.
数值结果展示了阶条件,算法的有效性和长时间性质.该方法不仅可使用文中提及
的辛格式,也可以使用其他合适的辛格式,对于非齐次的问题需要增加两个分量来求解.

2. 将复合变分准则应用到了 (2+1) 维拓展 QZK 方程.应用李点对称方法,
证明了 (2+1) 维拓展 QZK 方程是严格自共轭的,并且构造了其守恒律,给出了一维子代数
最优系统;通过相应的相似不变量的相似变换,将 (2+1) 维拓展 QZK 方程变为线性的偏微分方程.
研究结果表明李点对称方法对于拓展 QZK 方程是有效的,对研究该方程起到了积极的指导作用.

3. 将波形松弛方法和辛方法应用到哈密尔顿系统上,
提出了求解哈密尔顿系统的辛波形松弛方法,获得了很好的设计该格式的等式.该方法
结合了辛方法和波形松弛方法的优势,克服了求解隐式辛格式的困难,同时指明了针对该
系统选择分裂函数的方法,并使用窗口加速技术对求解过程进行加速.在辛波形松弛方法的
基础上,对李群方法进行了分析,讨论了隐式 RK-MK 方法的
必要性,并基于辛波形松弛方法的思路,对隐式的 RK-MK 方法提出了一种改进方法,即用波形
松弛方法进行修正.该方法能够简化隐式 RK-MK 方法的计算复杂性,把隐式计算简化为
较简单的显式或者半隐式问题,并使用窗口加速法进行加速.对连续问题和离散后问题,
分别证明了系统哈密尔顿量在迭代下收敛到守恒量,并给出了一个改进的李群 RK-MK
算法,还给出了算法的基本流程.数值结果对辛结构和李群
结构的保持说明了该类方法的可行性和有效性.

\section{展望}
\esection{Expectations}
在本博士论文的基础上,可以从如下几个方面进行进一步的深入研究.一是基于 RK-MK
方法,建立在流形上的保结构数值格式,这将给隐式 RK-MK 方法的使用提供理论依据;二是对
波形松弛方法进行修正和改进,使得数值结果收敛到相应数值格式这一问题变为提高数值格式
的精度;三是对辛波形松弛方法设计并行算法,使波形松弛可以用来设计并行算法的事实成为可能.
