\chapter{结论与展望}
\echapter{Conclusions and Expectations}
在本章, 将总结前面章节的主要内容, 并为下一步的研究工作进行展望.
\section{结论}
\esection{Conclusions}
本博士论文主要围绕辛方法,波形松弛方法和李群方法进行了一些数值方法和解析方法的研究.
通过对不守恒的系统进行变换得到了形式较好的守恒系统.通过对隐式格式的修正得到了较为
容易的显式格式.通过李点对称变换得到了形式较为简单的约化方程.具体内容包括以下几个方面:

(一) 在第~2~章,提出了一种新的求解带有齐次边界条件的电报方程的方法.该方法有效
利用了一个将非哈密尔顿系统化为哈密尔顿系统的变换,结合了辛方法的特性,得到了
较好的效果.本章中,讨论了该算法的阶条件、CFL 条件、长时间性质和局限性.在空间离散上
取了二阶的离散格式,得到了 $O(\Delta x^2+ \Delta t^k)$ 的误差界. 该方法的一个优点是
利用了辛方法长时间求解的好的性质.另外,我们的解可以看作一个很优美的哈密尔顿系统
乘以一个函数的结果.方法的基本思想是先变换,再求解,再逆变换,和傅立叶变换的思想
比较类似.数值结果展示了阶条件,算法的有效性和长时间性质.该方法不局限于使用文中提及
的辛格式,其他合适的辛格式也可以使用进来.非齐次的问题,需要增加两个分量来求解,该部分
在文中注解部分有所提及.

(二) 在第~3~章, 将复合变分准则应用到了 (2+1) 维拓展 QZK 方程.应用这些李对称,
证明了 (2+1) 维拓展 QZK 方程是自共轭的,并且构造了其守恒律.接着,给出了一维子代数
最优系统.通过相应的相似不变量的相似变换, (2+1) 维拓展 QZK 方程变为线性的偏微分方程.
说明了李点对称方法对于拓展 QZK 方程是有效的,该结果对研究该方程起到了积极的指导作用.

(三) 在第~4~章,对波形松弛方法和辛方法做了介绍,并将其应用到哈密尔顿系统上,
提出了求解哈密尔顿系统的辛波形松弛方法,得到了一个很好的设计该格式的等式.该方法
结合了辛方法和波形松弛方法双方的优势,解决了求解隐式辛格式的困难,同时指明了针对该
系统应该如何选择分裂函数的问题.使用窗口加速技术来加速该方法.在辛波形松弛方法的
基础上,对李群方法做出了详细地介绍,分析了其优点和不足,分析了隐式 RK-MK 方法的
必要性,并基于辛波形松弛方法的思路,对隐式的 RK-MK 方法提出了一种改进方法,即用波形
松弛方法进行修正.该方法能够缓解隐式 RK-MK 方法的复杂性计算问题,把隐式的计算化为
较为简单的显式或者半隐式问题,并再次使用窗口计算进行加速.从连续问题和离散后问题
分别给出了系统哈密尔顿量在迭代下收敛到守恒的量的性质,并给出了一个改进的李群 RK-MK
算法,还给出了算法的基本流程.在给出的数值结果中,能看到较好的对辛结构和李群
结构的保持,从而说明了该类方法的可行性和有效性.

\section{展望}
\esection{Expectations}
在本博士论文的基础上,可以从如下几个方面进行进一步的深入研究.一是基于 RK-MK
方法,建立在流形上的保结构数值格式,这将给隐式 RK-MK 方法的使用提供理论依据.二是对
波形松弛方法进行修正和改进,使得数值结果收敛到相应数值格式这一问题变为提高数值格式
的精度.三是对辛波形松弛方法设计并行算法,使波形松弛可以用来设计并行算法的事实成为可能.
