\chapter{绪论}
\echapter{Preface}

\section{研究背景和现状}
\esection{Research Background and State of the Art}

随着计算机软硬件的迅猛发展,计算机形态的日新月异,机器人科学和量子计算机的出现与
蓬勃发展,对高效高性能算法的研究变得越来越重要.高性能的计算方法在工业生产中
起到了决定性的核心作用.算法的计算速度快,实现简单,性能稳定和可并行性已经成为许多领域里
的基本需求,拥有这些算法也就意味着拥有了国家的核心竞争力.在数学领域里,为工程应用中提供
这些高效高性能的算法,其现实意义和重要性是十分明显的.在工程应用中,我们关注这样一类特殊的问题,此类
问题对应具有守恒律的系统,因此对计算效果有特殊的要求.然而,方程的守恒律在常规的数值计算格式
中往往无法直接体现.可见,对于一个本身具有守恒性质的微分方程,只有合理地设计数值格式,才能达到
相应的计算要求.

本论文通过对与方程守恒性质相关的一些高性能算法(包括具有守恒性质
或者近似有守恒性质微分方程的辛方法,辛波形松弛方法),李群方法以及微分方程的李点对称方法展开研究,进行改进与创新,获得了稳定的易于实现的数值
方法以及微分方程的一些守恒性质.所涉及的方向包括电路模拟,哈密尔顿系统的计算,流形上
的数值计算以及偏微分方程的约化和解析解.

\subsection{电报方程}
\esubsection{Telegraph Equations}

传输线主要用来传输无线电频率的交变电流,反映了当电流频率高到一定程度时的波的性质.
传输线广泛应用在信号传输,产生脉冲和信号过滤等领域.常见的传输线主要有平行线、同轴电缆、
带状线、微带线、波导管、介质波导和光纤,等等.电报方程,也称为传输线方程或电报员方程,
来源于Maxwell方程组,是描述传输线中电流电压属性的一类方程.更具体地讲,电报方程来源于这样的问题,
导体由无限个二端口元件连接而成,每一个元件都是很小的传输线元件.

现有的关于电报方程的求解方法通常分为两大类,第一类是求精确解,第二类是求数值解.精确解只对一少部分
方程有效,其他问题我们还需要用数值模拟来解决.求数值解的方法,从思想上可以分为两种,第一种是
用解析的方法,将解的波形迭代起来求解,第二种则是离散方程之后的数值计算,即离散方法.解析方法通过迭代波形,
使得结果逐渐逼近真解.常见的解析方法有 Exp-函数方法~\cite{naher2011exp}、 Adomian 分解
方法~(ADM)~\cite{adomian1988areview,sheikholeslami2012analytical}、 变分迭代方
法~(VIM)~\cite{wu2013variational} 和同伦摄动方法~(HPM)~\cite{sheikholeslami2012homotopy}. 电报
方程的离散方法常见的有微分二次方法~(DQM)~\cite{jiwari2012numerical}、 交替迭代隐式
方法~(ADI) \cite{cui2013convergence} 和 WENO 方法 \cite{borges2008improved,shen2014improvement}.
电报方程的系统能量随时间而指数衰减,不具有能量守恒律.经过研究,我们发现了这样一个变换 \cite{polyanin2001handbook}, 能够使电报方程变为
一个守恒的系统,而这种守恒性刚好能让我们利用辛方法进行求解.这样做的好处是能够有效地
求解更长的时间区间,同时能够获得更容易实现的数值格式.

\subsection{拓展 QZK 方程}
\esubsection{Extended QZK Equations}

拓展 quantum Zakharov-Kuznetsov~(QZK) 方程描述了等离子体里的物理现象.对于拓展 QZK 方程的
研究追溯到若干年前, Zakharov 和 Kuznetsov \cite{abdou2011quant} 构造了描述由冷离子和热电子构成的
磁等离子体的非线性离子波 (IAWs) 的方程.量子等离子体及其它们的性质获得了越来越多的理论
物理和实验物理研究者的注意,这是由于它们在德布罗伊波长超过德拜波长接近费米波长时的带电载体特性值得研究 \cite{abdou2011quant,ahmed2013kinks,bhrawy2013soli,biswas20091soli,biswas2013soli,bluman2010appli,elganaini2011tra,godleswski2004the,guner2015bright,ibragimov2006inte}. 在均匀磁场里的弱非线性离子波的性质由量子 Zakharov-Kuznetsov 方程来刻画.这些年来,许多
研究者对该问题的关注越来越多,在不同的量子等离子体模型中研究了磁场的影响 \cite{ibragimov2007anew,iwasaki1990cylin,johnpilai2011sym,khan2008linear,krishnan2010sol,leveque1992num,linares2009well,linares2011local,morris2013soli,moslem2007soli,mothibi2015con,moussa2001simi,munro2014con,munro2000sta,mushtaq2005non,olver2000app,peng2008exact,sabry2009non}.

(2+1) 维 Zakharov-Kuznetsov (ZK) 方程的主要研究手段有正弦余弦方法、拓展双曲正切方法、
同伦分析方法 \cite{linares2009well}、 简化的 Hirota 方法 \cite{biswas2013soli,bluman2010appli}
和映射方法 \cite{morris2013soli}, 等等.带有非线性扩散项和时间相关系数的 (2+1) 维广义
Zakharov-Kuznetsov (GZK) 方程由孤立波拟设方法所研究 \cite{sabry2009non}. (3+1) 维 QZK
方程可以通过辅助方程方法 \cite{ahmed2013kinks} 和拓展 F 展开 (EFE) 方法得到 \cite{munro2000sta}.
在 \cite{ahmed2013kinks} 中作者使用了约化摄动方法正式地得到了拓展 quantum Zakharov-Kuznetsov (extended QZK) 方程
,该方程由广义拓展方法 \cite{guner2015bright}, Jacobi 椭圆正弦和余弦函数 \cite{biswas20091soli}
所研究.同时,李点对称方法和极简方程方法也可以用来研究 Zakharov-Kuznetsov 方程 \cite{leveque1992num}.
此外,对于一类广义 (2+1) 维 Zakharov-Kuznetsov 方程也有类似的工作 \cite{moslem2007soli}.

\subsection{哈密尔顿系统}
\esubsection{Hamiltonian Systems}

哈密尔顿系统是由数学家 Hamilton 构造出来,用来描述物理系统的发展型方程.该系统
的表示巧妙地引入动量的概念,将一个二阶问题转化为维数加倍的一阶问题.该表示
的优点在于能够在无法求得该系统的解析解的情况下,阐明一些动力系统的性态.在工程应用中,包括
动力学、分子动力学、流体力学、量子力学、图像处理、天体力学和核工程等领域在内的问题,都可以表示成
相应的哈密尔顿系统的形式 \cite{arieh2009afirst}.

对于哈密尔顿系统,有一类比较成熟的、适于计算较长时间区间的方法,称为辛方法 \cite{feng2010symplectic}. 辛方法求解
较长时间区间的哈密尔顿系统著称,用辛方法求解哈密尔顿系统给予人们新的启迪,指引着人们去寻找好的优美的方法.辛方法起源于
de Vogelaere (1956), Ruth (1983)和 冯康 (1985) 的工作 \cite{hairer2006geometric}, 其基本
思想是避开盲目对精度的要求,巧妙地利用哈密尔顿系统的辛结构,改为以保持该结构作为目标,构造
数值格式.近些年的理论和数值算例都验证了该方法的优越性,系统的数值能量能够保持在系统的
真实值附近波动.

数十年来,辛方法迅速发展起来,并且得到了不断地发扬光大 \cite{calvo1994numerical,leimkuhler2004simulating,hong2006multi,yang2009extended,monovasilis2013exponentially,xin2016birkhoffian,michalas2016numerical,liao2016multi}. 辛方法数值结果证实了辛数值积分格式比非辛格式的优越性,因此辛方法在处理多体问题和
其他哈密尔顿系统问题上,有着较好的应用.常用的辛方法有辛欧拉方法,辛 Runge-Kutta 方法,
辛 Runge-Kutta-Nystr{\"o}m 方法 \cite{kalogiratou2014fourth,kalogiratou2015}, 辛 ERKN
方法 \cite{wang2014ahigh}, 哈密尔顿 BVM \cite{brugnano2014multi}, 等等.此外,
关于哈密尔顿系统的研究还有两步波形松弛方法 \cite{hassanzadeh2014two}, 针对非自治
哈密尔顿系统的方法 \cite{hong2000numerical,zhang2010anote}, 随机哈密尔顿方程的
方法 \cite{burrage2014structure,ma2015sto,fan2015using}, 多辛方法 \cite{wang2013multi} 和
最有控制系统的研究 \cite{li2015asym}, 等等.这些方法都直接或间接地利用了辛方法的特性,构造
出了较好的数值格式.我们还知道,辛 Runge-Kutta 方法均为隐式格式 \cite{sanz1988runge}, 即
没有显式的保辛的 Runge-Kutta 法.

\section{主要方法}
\esection{Main Methods}
在本小节,我们对本文中所用到的基本方法作简单介绍与总结,这些方法包括辛方法,李群方法和波形松弛方法.

\subsection{辛方法}
\esubsection{The Symplectic Method}

辛方法是求解哈密尔顿系统的一类特殊的数值方法,该方法结合哈密尔顿系统的辛结构而提出,在计算过程中有着优秀的特性.

辛方法的定义是基于哈密尔顿系统的辛结构给出的.为了在数值计算上得到很好的效果,辛格式要求在
每个数值时间点上保持 $q$ 与 $p$ 的外微分二形式 $dq_{n+1}\wedge dp_{n+1}=dq_n\wedge dp_n$ 不变, 这里含有下指标的 $q$ 和 $p$
分别是在不同离散时间点上的广义位移和广义动量.在设计数值格式时,为了使验证变得容易,我们采用一个与之等价的易于进行验证
的定义,即对于一个单步的数值方法 $z_{n+1}=\Phi_h(z_n)$, 保持 $(\nabla\Phi_h)^TJ^{-1}(\nabla\Phi_h)=J^{-1}$.

辛格式的数值格式有很多,最常用的是隐式中点法,辛欧拉法, St\"{o}rmer-Verlet 方法,辛 Runge-Kutta 方法,等等.
构造辛格式的方法也有很多,辛 Runge-Kutta 方法就是其中一大类,还有基于生成函数的方法,多级单步格式,等等.
并且基于保辛格式和保结构的数值方法的研究一直在不断推进.

关于辛方法的理论研究主要有辛 Runge-Kutta 方法的阶数分析,包括根树理论和 B 级数理论;保辛和保能量
的研究及其相互之间的关系研究;反相误差分析理论;哈密尔顿摄动理论;高震荡方程的辛方法;多步法理论;
数值格式的长时间性态理论,等等.

\subsection{李群方法}
\esubsection{The Lie Group Method}

李群 \cite{wanner1983fun} 是包含群结构的微分流形.作为群,李群有两个代数运算,乘法和逆.同时作为
微分流形我们要求李群的两个代数运算是光滑映射.李群对应的李代数是李群在单位元处的切空间,
作为一个线性空间,李代数上包含一个双线性映射,称为李括号,满足反对称交换律和 Jacobi 恒等式.

指数映射是连接李群和李代数的纽带,其定义为 $\mathrm{exp}: \mathfrak{g}\to \mathcal{G}$,
式中 $\mathfrak{g}=T\mathcal{G}$ 为 $\mathcal{G}$ 的切空间.这里我们要求有如下关系成立 $\mathrm{exp}(x)\mathrm{exp}(y)=\mathrm{exp}(xy)$.
矩阵李群是李群的一个特例,其群元素均为矩阵,对应的指数映射即为矩阵的指数函数
\begin{equation*}
	\mbox{expm}(A)=\sum_{j=0}^{\infty}\frac{A^j}{j!}.
\end{equation*}}
指数映射由于其形式和指数函数的定义相同而得名.

本文中的李群方法是指使用李群作为工具的一系列方法的统称,主要包含流形上的李群方法和李点对称方法.

\subsubsection*{\textbf{流形上的李群方法}}

本文中我们研究的李群方程是指在流形上的矩阵方程
\begin{equation*}
	Y'=A(t,Y)Y,\quad t\geq 0,\quad Y(0)=Y_0,
\end{equation*}
式中 $Y_0\in G $, $A:\mathbb{R}^+\times G\to \mathfrak{g}$, 这里的 $Y$ 和 $A(t,Y)$ 都是时间相关的 $N\times N$ 的矩阵.

对于流形上的李群方程,我们知道初始值在流形上的解的流随着时间依然保持在流形上.
通常来说,我们的数值计算都是在一个空间上进行的,常用的空间是平直的线性欧氏空间.
然而,有些方程具有特殊的意义,这些问题的解在一个流形上演化.在全空间上数值计算的
误差存在会导致计算结果并不在该流形上.为了解决这一类问题,有一类流形上的李群方法被
设计出来,这些方法能够很好地保证数值解落在流形上. Runge-Kutta Munthe-Kaas(RK-MK) 算法就是这样一类方法.李群
方法起初是由 Crouch 和 Grossman \cite{crouch1993numerical} 提出来的,即所谓
的 Crouch-Grossman 方法,该方法得到了一系列地推广
\cite{faleinsen2001multi,zaletkin2010numerical,bulychev2001numerical,buono2003numerical,billo1992numerical}.
后来,在 1996 到 1999 年, Munthe-Kaas  等人提出了该方法的一个改进方法
\cite{mk1996lie,mk1997numerical,mk1998runge,mk1999high}, 即 MK-RK 方法,以及随后做的一些推广 \cite{ostermann2010exponential,owren2000the,bruls2012lie,munthe2013onpost,garcla2011onalg}.
Crouch-Grossman 方法和 MK-RK 方法都用到了李群的指数映射,联结了李群这个流形和
李代数,并且都利用了 Runge-Kutta 方法这种数值结构,使得阶数可以设计得很高.
而且常规意义下的李群方法均为显式方法,计算较为简便.

\subsubsection*{\textbf{李点对称方法}}

李点对称方法 \cite{olver2000app} 是基于单参数李群,并将李群作用在微分方程上的一类方法.起初,该方法根据
李的三个定理,用来把常微分方程从高阶化成较低阶,或者求出一些特解.后来逐渐演变为对偏微分方程的约化,
从较高阶化为较低阶,或者从多元化为一元,等等.该方法是一种求解微分方程特解
或者约化微分方程的方法.其思想是将微分方程放到 Jet 空间中研究,该空间的自变量是原方程的自变量,因
变量和因变量的各阶导数或偏导数.将李群作用到该空间,我们通过某种设计变换的模式,可以程式化地得到
单参数李群,据此经过一系列计算,我们得到一些关于方程的性质或者方程的解.

本文中,我们基于文献 \cite{sjoberg2007dou} 进一步研究拓展 QZK 方程.守恒律在研究微分
方程中扮演了重要角色,因为守恒律往往对应着物理中的守恒律,比如质量守恒、能量守恒、
动量守恒、角动量守恒、电荷守恒或其他运动的约束 \cite{mushtaq2005non,song2013top,song2013dom}.
守恒律可以用来研究非线性偏微分方程解的存在唯一性和解的稳定性 \cite{wang2014soli}, 此外
也有关于数值解的研究 \cite{wazwaz2005exact,wazwaz2008the}. 因此,我们对拓展 QZK 方程研究了
偏微分方程的守恒性质,证明了 (2+1) 维拓展 QZK 方程是严格自共轭的,并且构造了其守恒律.
接着,我们给出了一维子代数最优系统.通过相应的相似不变量的相似变换, (2+1) 维拓展 QZK 方程
变为线性的偏微分方程.说明了李点对称方法对于拓展 QZK 方程是有效的,该结果对研究该方程
起到了积极的指导作用.

\subsection{波形松弛方法}
\esubsection{The WR Method}

波形松弛方法来源于对大规模集成电路系统的研究 \cite{lelarasmee1982waveform}. 在电路分析中,我们知道,
有许多电路定理和等效电路定理,可以将大规模电路转化成若干个等效电路.基于这个思想,有了波形松弛方法
的雏形,即分裂电路系统,迭代求解的过程.波形松弛方法主要分为两个部分,选取分裂函数对系统分裂和对分裂后的系统迭代
 \cite{jiang2009wr,burrage1995parallel,jacob1985waveform}. 首先,需要使用分裂函数
 将规模大的系统划分为较小规模的子系统,这些分裂之后的子系统之间可以是解耦的,因此可能进行
 并行求解.最后是迭代,迭代的过程先分别求解子系统,再在系统之间交换信息,进行下一次迭代,
直到达到中止条件.由此可见,波形松弛方法的显著特性是问题解耦和潜在的可并行性.波形松弛方法给我们
指出了一种不同的求解微分方程的方式.它不是简简单单的通过求解离散过后的代数方程组的
松弛方法,而是通过对方程分裂而构造出来的一种函数迭代方法,是``波形''的迭代.波形松弛方法的
一个优点是能够把耦合的系统,通过分裂,在迭代过程中化为非耦合的系统,即解耦.对于一些问题
来讲,这为算法的并行计算提供了条件.此外,注意到,在计算的过程中,波形松弛方法能够
将一些只能隐式格式求解的问题,化为显式或者半隐式的格式即可求解的问题.

简而言之,波形松弛方法 \cite{jiang2009wr} 作为一个数学工具,思想起源于大规模集成电路的
求解.当系统的规模大到一定程度的时候,使用现有的计算资源求解该问题就变得困难,
可以把大型系统分裂成更易于求解的小型系统来做.为了达到这个目标,选取分裂函数 $F(z,y)$ 使
得 $F(z,z)=f(z)$, 然后将方程中的 $f(z)$ 替换为 $F(z,y)$. 接下来为每一个子
系统选择一个合适的初始迭代,我们将 $F(z,y)$ 中的 $y$ 作为已知函数, $z$ 作为未知函数,这样
每次迭代的计算量就降低了.然而这样的一步计算不会得到系统的真实解,这可以由初值的选取和格式
的看出来.因此要通过迭代来保证收敛到方程的真解.通常将其
记作 $\dot{z}^{k+1}=F(z^{k+1},z^{k})$, 式中 $k=0,1,\cdots$ 是迭代次数.

\section{论文的主要内容和组织结构}
\esection{Main Results and Organizations of the Dissertation}

本文主要涉及了三部分内容,研究了一类电报方程的辛方法,提出了辛波形松弛方法,流形上改进的李群方法
并利用李点对称对偏微分方程进行了约化,主要研究内容和创新点如下:

(一) 提出了一种新的求解带有齐次边界条件的电报方程的方法.该方法巧妙地利用了一个将非哈密尔顿系统化为哈密尔顿系统的变换,结合了辛方法的特性,得到了较好的效果.本章中,我们讨论了该算法的阶条件,~CFL 条件,长时间性质和局限性.在空间离散上我们取了二阶的离散格式,得到了 $O(\Delta x^2+ \Delta
t^k)$ 的误差界. 该方法的一个优点是利用了辛方法长时间求解的好的性质.另外, 我们的解可以看作一个很优美的哈密尔顿系统乘以一个函数的结果.我们方法的基本思想是先变换,再求解,再逆变换,和傅立叶变换的思想比较类似.数值结果展示了阶条件,算法的有效性和长时间性质.该方法不局限于使用文中提及的辛格式,其他合适的辛格式也可以使用进来.非齐次的问题,需要增加两个分量来求解,该部分在文中注解部分有所提及.

(二) 将复合变分准则应用到了 (2+1) 维拓展 QZK 方程.应用这些李点对称,证明了 (2+1) 维拓展 QZK 方程是严格自共轭的,并且构造了其守恒律.接着,给出了一维子代数最优系统.最后,通过相应的相似不变量的相似变换,将 (2+1) 维拓展 QZK 方程约化为线性的偏微分方程.

(三) 提出了求解哈密尔顿系统的辛波形松弛方法.该方法给出了针对哈密尔顿系统应该如何选择分裂函数的一个可行性建议.使用窗口加速技术来加速该方法,给出了系统哈密尔顿量在迭代下收敛到守恒的量的性质,并在数值结果中验证了这一点.在此基础上,对李群方法做出了讨论,分析了其优点和不足,并对隐式的 RK-MK 方法提出了一种改进方法,用波形松弛方法进行修正.该方法能够缓解隐式 RK-MK 方法的复杂性计算问题,把隐式的计算化为较为简单的显式或者半隐式问题,同时利用窗口计算进行加速.还给出了算法的基本流程,通过数值结果说明了该方法的有效性.

在本文的最后,总结概括了本文的创新点,并对进一步的研究工作进行了展望.
